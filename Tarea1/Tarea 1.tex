\documentclass[12pt,dvipsnames]{article}
\usepackage[letterpaper, left=2.54cm, right=2.54cm, top=2.54cm, bottom=2.54cm]{geometry}
\usepackage[spanish]{babel} % Establece el idioma en Español
\usepackage[utf8]{inputenc} % Permite el uso de carácteres
\usepackage[T1]{fontenc} % Para poder copiar texto de word o pdf sin problemas
\usepackage{datetime}
\usepackage[T1]{fontenc}
\newcommand{\mydate}{\formatdate{23}{3}{2020}}
\usepackage{fancyhdr}
\usepackage{tikz}
\usetikzlibrary{decorations.pathreplacing}
\usetikzlibrary{intersections}
\usetikzlibrary{decorations.pathreplacing}
\usepackage{etoolbox}
\AtBeginEnvironment{tikzpicture}{\shorthandoff{>}\shorthandoff{<}}{}{}
\fancyhf{}
\setlength{\parindent}{0pt} %Quitar sangría


\long\def\mytitle{%
	\begin{titlepage}
		\begin{flushleft}
			\begin{minipage}{0.2\textwidth}%
				\includegraphics[width=2\textwidth]{logo2.png}%
			\end{minipage}\hspace{30pt}
			\begin{minipage}{0.7\textwidth}%
				\textsc{\color{black}%
					Pontificia Universidad Católica de Chile\\
					Facultad de Agronomía e Ingeniería Forestal\\
					Departamento de Economía Agraria y Ambiental}\\
					%&Fundamentos de la Economía Agraria y Ambiental}\\%%%%% Cambiar según curso
				
				\begin{tikzpicture}%
				\draw[thick, black] (0,0)--(0.99\textwidth,0);%   
				\end{tikzpicture}%
			\end{minipage}%
		\end{flushleft}
		\vspace{1cm}
		{\centering
			\vspace{7cm}
			{\huge\bfseries Tarea 1: Repaso de Conceptos Microeconómicos Básicos\par}}
		{\centering
			\vspace{5 cm}
			{\large\itshape Nombre: Felipe Burry\par}
			{\large\itshape Profesor: William Foster\par}
			{\large\itshape Curso:  AGE3811 Fundamentos de la Economía Agraria y Ambiental\par}
			\vspace{1 cm}
			{\large \mydate\par}
		}
	\end{titlepage}
}


%%%%%
%%%%% Texto

\begin{document}
	
	\mytitle \newpage
	
1. Consideremos dos “países” (o personas, regiones, etc.) con la posibilidad de producir durante el año dos productos, grano, G, y fruta, F. Supongamos que los países tienen las siguientes fronteras de producción (restricciones presupuestarias) dado por:\\

País\#1: $$F = 30 - G$$\\
País\#2: $$F = 20 - \frac{1}{4}G$$\\

También considere que las preferencias de los dos países son tal que los dos bienes son usados/consumidos en una proporción 1 a 1; es decir, $G = F$. (Este tipo específico se llama “complementarios perfectos”.) Se pide:\\

a.	Grafique las fronteras de producción para país \#1 y país \#2. (10 puntos)\\ \\

\begin{minipage}{.5\textwidth}
	\centering
		
	\begin{tikzpicture}[scale=0.5]
	\draw[thick,<->] (0,8) node[above]{$F$}--(0,0)--(10,0) node[right]{$G$}; % Axis and Lable
	\draw(0,6)--(6,0) node[right]{$$}; %30 30
	\node[left]at (0,6){$30$}; 
	\node[centered] at (6,-0.4){$30$}; 
	\node[above,font=\normalsize] at (current bounding box.north) {Frontera de Producción del País \#1};
	\end{tikzpicture} 
	
\end{minipage}%
\begin{minipage}{.5\textwidth}
	\centering
	\begin{tikzpicture}[scale=0.5]
	\draw[thick,<->] (0,8) node[above]{$F$}--(0,0)--(10,0) node[right]{$G$}; % Axis and Lable
	\draw(0,4)--(9,0) node[right]{$$}; %20 80
	\node[left]at (0,4){$20$}; 
	\node[centered]at (9,-0.4){$80$}; 
	\node[above,font=\normalsize] at (current bounding box.north) {Frontera de Producción del País \#2};
	\end{tikzpicture}
\end{minipage}
\\
\\
b. ¿Quién tiene la ventaja absoluta de producir grano? ¿Quién tiene la ventaja absoluta de producir fruta? (5 puntos)\\

La \underline{ventaja absoluta} en la producción de Granos la tiene el país \#2, ya que puede producir 80 unidades de Grano en un año. La \underline{ventaja absoluta} en la producción de Frutas la tiene el país \#1, ya que puede producir 30 unidades de Fruta en un año

\begin{table}[h]
	\centering
	\begin{tabular}{|cccc|}
		\hline
		\multicolumn{2}{|c|}{\textbf{País \#1}}                                        & \multicolumn{2}{c|}{\textbf{País \#2}}                    \\ \hline
		\textit{\textbf{Grano (G)}} & \multicolumn{1}{c|}{\textit{\textbf{Fruta (F)}}} & \textit{\textbf{Grano (G)}} & \textit{\textbf{Fruta (F)}} \\ \hline
		0                           & \multicolumn{1}{c|}{30}                          & 0                           & 20                          \\
		30                          & \multicolumn{1}{c|}{0}                           & 80                          & 0                           \\ \hline
	\end{tabular}
	%\caption{Tabla}
	
\end{table}


\newpage
c.	¿Quién tiene la ventaja comparativa de producir grano? ¿Quién tiene la ventaja comparativa de producir fruta? ¿Cuál es el costo de oportunidad de producir una unidad más de grano en país \#1 en términos de fruta? ¿Cuál es el costo de oportunidad de producir una unidad más de grano en país \#2 en términos de fruta? ¿El costo de oportunidad de producir una unidad más de fruta en país \#1 en términos de unidades de grano? ¿El costo de oportunidad de producir una unidad más de fruta en país \#2 en términos de unidades de grano? (5 puntos) \\

Dado que: 

\begin{itemize}
	\item Para el país \#1 $30G = 30F$ se deduce que $G = F$ y $F = G$
	\item Para el país \#2 $80G = 20F$ se deduce que $G = \frac{1}{4}F$ y $F = 4G$
\end{itemize}

Con esto se sabe que la \underline{ventaja comparativa} de producir Grano la tiene el país \#2 ya que su costo de oportunidad es de solo $\frac{1}{4}$ de Fruta en comparación con el país \#1 que le cuesta $1$ de Fruta.
Del mismo modo se sabe que la ventaja comparativa de producir Fruta la tiene el país \#1 ya que se costo de oportunidad es de $1$ de Grano en comparación con el país \#2 que le cuesta $4$ de Grano.\\

En lo que respecta a los \underline{costos de oportunidad} se tiene que:
\begin{itemize}
	\item Costo de oportunidad de $1$ unidad más de Grano en términos de Fruta para el país \#1 = 1 Fruta
	\item Costo de oportunidad de $1$ unidad más de Grano en términos de Fruta para el país \#2 = $\frac{1}{4}$ Fruta
	\item Costo de oportunidad de $1$ unidad más de Fruta en términos de Granos para el país \#1 = 1 Grano
	\item Costo de oportunidad de $1$ unidad más de Fruta en términos de Granos para el país \#2 = 4 Granos
\end{itemize}

d.	¿Cuáles serían las canastas de consumo óptimas para los dos países? En una situación sin intercambio/comercio – es decir, en el estado de autarquía. (10 puntos)\\

Dado que los bienes son complementos perfectos, es decir, se consumen en una proporción 1:1 sabemos que se debe buscar para cada país una relación tal que F = G. De esta forma para el País \#1 se tiene que $F = 30 - G = G$, por tanto, $G = 15 = F$. Del mismo podo para el País \#2 se tiene que $F = 20 - \frac{1}{4}G = G$, por tanto, $G = 16 = F$. Así, la canasta de consumo óptima para cada país en estado de autarquía es


\begin{table}[h]
	\centering	
	\begin{tabular}{ccc}
	                  & \textbf{Grano (G)} & \textbf{Fruta (F)} \\ \hline
	\textbf{País \#1} & 15                 & 15                 \\
	\textbf{País \#2} & 16                 & 16                 \\ \hline
	\textbf{Total}    & 31                 & 31                
\end{tabular}
%\caption{Canasta de consumo óptima en estado de autarquía}

\end{table}
\newpage

e. Usando fruta como el bien numerario $P_F=1$, encuentre un precio de grano, $P_G$, tal que las dos personas pueden intercambiar/comerciar para el beneficio de ambos.  (10 puntos)\\

Sabiendo que el Costo de Oportunidad del Grano se puede escribir en términos de Frutas, podemos obtener el \underline{Precio} del Grano ($P_G$) como sigue:
$$P_G = \frac{F}{G}*P_F$$
Y dado que $P_F=1$ se tiene que 
$$P_G=\frac{F}{G}$$
Además si consideramos la máxima producción de fruta posible entre los países se tiene que
$$F_t = 50 - \frac{1}{4}G \: ; \: 0 < G \leq 80$$
$$F_t = 110 - G \: ; \: 80 < G \leq 110$$
Y dado que el consumo es $F = G$ se tiene que
$$50 - \frac{1}{4}G = G$$
$$G = 40 \: ; \: F = 40$$
Y ya con esto podemos deducir que
$$P_G = 1$$

f.	Muestre una transacción posible entre los dos tal que los dos ganan en comparación al caso de autarquía. Es decir, que los dos consumen en una proporción 1 a 1 y los niveles totales de producción en los dos países de los dos bienes es suficiente para satisfacer los niveles totales de consumo. (20 puntos)	\\

Considerando que en el País \#2 el costo de oportunidad de producir Fruta es de 4 Granos, en el País \#1 el costo de oportunidad de producir Fruta es de sólo 1 Grano, y  el $P_F = 1$, para que exista comercio se debe buscar un precio tal que $1 < P_t \leq 4$. Para la simplificación de los cálculos se usará un valor de $P_t = 2$ y que por la ventaja comparativa el País \#1 se dedicará exclusivamente a la producción de Frutas (es decir, producir 30 Frutas).\\
Si se considera la canasta de consumo donde $F = 40$, al País \#2 le correspondería producir $40 -30 = 40$ Frutas de forma que
$$40 - 30 = 30 - \frac{1}{4}G$$
$$G = 40$$ 
Por tanto, las nuevas condiciones de consumo con comercio son:\\
\begin{table}[h]
	\centering	
	\begin{tabular}{ccc}
		& \textbf{Grano (G)} & \textbf{Fruta (F)} \\ \hline
		\textbf{País \#1} & 0                 & 30                 \\
		\textbf{País \#2} & 40                 & 10                 \\ \hline
		\textbf{Total}    &   40               & 40                
	\end{tabular}
	%\caption{Canasta de consumo óptima en estado de autarquía}
\end{table}\\

De esta forma ambos países están aumentando su consumo de Fruta y Grano por lo que con el comercio ambos países mejoran su Utilidad.\\

2. Considere una curva de oferta y demanda en el mercado diario de harina en un pequeño pueblo:
$$p_s=\frac{1}{5} q^s$$
$$p_d=2600-5q^d$$
Donde el precio está en \$/Kg y la cantidad en Kilogramos.\\

A.	En la ausencia de impuestos u otras restricciones, en el equilibrio:

\begin{minipage}{.5\textwidth}
	\centering
	
	\begin{tikzpicture}[scale=0.5]
	
	\draw[thick,<->] (0,10) node[above]{$P$}--(0,0)--(10,0) node[right]{$Q$}; % Axis and Lable
	\node [below left] at (0,0) {$o$};%Origin
	\draw[red](1,1)--(9,9) node[right]{$S$};
	\draw[blue](1,9)--(9,1) node[right]{$D$};
	\draw[dashed, black!30!green](0,5)--(5,5) node[right]{$$};
	\draw[dashed, black!30!green]((5,0)--(5,5) node[right]{$$};
	\filldraw[fill=black!30!green] (5,5) circle ( 5pt) {};
	
	%Textos
	\node[left]at (0,5){$100$}; %Texto en eje y en linea de equilibrio
	\node[centered]at (5,-0.75){$500$}; %Texto en eje x en linea de equilibrio
	\end{tikzpicture}
	
	\end{minipage}%
	\begin{minipage}{.5\textwidth}
	\centering
	\begin{eqnarray*}
		\textnormal{Dado que en el equilibrio} & & p_s = p_d\\
		\textnormal{Se tiene que} & & \frac{1}{5} q^s = 2600-5q^d\\
		\textnormal{Por lo que} & & q^* = 500\\
		\textnormal{y} & & p^* = 100\\
	\end{eqnarray*}
	\end{minipage}

\begin{itemize}
	\item ¿Cuál es la cantidad del producto vendido y a qué precio? 
\end{itemize}	
Se venden 500 unidades a un precio de \$100 cada una

\begin{itemize}
	\item ¿Cuál es el ingreso total de la venta del producto?
\end{itemize}	
$Ingreso Total (I) = P*Q$ por lo que el Ingreso Total es de $\$50.000$\newpage

B.	Si el gobierno impone un precio piso de \$125/kg:\\\\

\begin{minipage}{.5\textwidth}
	\centering
	\begin{tikzpicture}[scale=0.5]
	\draw[thick,<->] (0,10) node[above]{$P$}--(0,0)--(10,0) node[right]{$Q$}; % Axis and Lable
	\node [below left] at (0,0) {$o$};%Origin
	\draw[red](1,1)--(9,9) node[right]{$S$};
	\draw[blue](1,9)--(9,1) node[right]{$D$};
	\draw[black!30!green](0,5)--(5,5) node[right]{$$};
	%\draw[black!30!green]((5,0)--(5,5) node[right]{$$};
	\filldraw[fill=black!30!green] (5,5) circle ( 5pt) {};
	\draw[dashed, black](0,6)--(6, 6) node[right]{$$};
	\draw[dashed, black](4,-1)--(4, 6) node[right]{$$};
	\draw[dashed, black](6,-1)--(6, 6) node[right]{$$};
	\node[left]at (0,5){$100$}; %Texto en eje y en linea de equilibrio
	\node[left]at (0,6){$125$}; %Texto en eje y en linea de equilibrio
%	\node[centered]at (5,-0.75){$500$}; %Texto en eje x en linea de equilibrio
	\node[centered]at (4,-1.5){$495$};
	\node[centered]at (6,-1.5){$625$};
    \draw[thick, decorate,decoration={brace, amplitude=12pt}, xshift=0pt, yshift =0pt] (4,6) -- (6,6) node[ black, midway, above, yshift=0.5cm]{Exedente};
	\end{tikzpicture}
	\end{minipage}%
	\begin{minipage}{.5\textwidth}
	\centering
	\begin{eqnarray*}
	\textnormal{Dado que } & & 125=\frac{1}{5} q^s_t\\
	\textnormal{Por tanto} & &  q^s_t = 625\\
	\textnormal{Del mismo modo} & & 125=2600-5q^d_t\\
	\textnormal{y} & & q^d_t = 495\\
	\end{eqnarray*}
	\end{minipage}%

\begin{itemize}
	\item ¿Cuántos kilogramos serán vendidos? 
\end{itemize}
Se venderán 495 Kg

\begin{itemize}
	\item ¿Es un equilibrio sostenible? Explique
\end{itemize}

No, ya que existe un exceso de oferta (excedente) porque los productores tienen los incentivos para producir más de lo que está siendo demandado producto del mensaje erroneo que envía el precio intervenido.  Pero a la larga la producción disminuiría y tendería a equilibrarse (siguiendo la lógica de Smith) ya sea en el punto de equilibrio o en las 495 unidades que el mercado demanda a este precio \\

C.	Si el gobierno impone un precio techo de \$90/kg:

\begin{minipage}{.5\textwidth}
\centering
\begin{tikzpicture}[scale=0.5]
\draw[thick,<->] (0,10) node[above]{$P$}--(0,0)--(10,0) node[right]{$Q$}; % Axis and Lable
\node [below left] at (0,0) {$o$};%Origin
\draw[red](1,1)--(9,9) node[right]{$S$};
\draw[blue](1,9)--(9,1) node[right]{$D$};
\draw[dashed, black!30!green](0,5)--(5,5) node[right]{$$};
%\draw[dashed, black!30!green]((5,0)--(5,5) node[right]{$$};
\filldraw[fill=black!30!green] (5,5) circle ( 5pt) {};
\draw[dashed, black](0,4)--(6, 4) node[right]{$$};
\draw[dashed,black](4,-1)--(4, 4) node[right]{$$};
\draw[dashed,black](6,-1)--(6, 4) node[right]{$$};
\node[left]at (0,5){$100$}; %Texto en eje y en linea de equilibrio
\node[left]at (0,4){$90$}; %Texto en eje y en linea de equilibrio
%\node[centered]at (5,-0.75){$500$}; %Texto en eje x en linea de equilibrio
\node[centered]at (4,-1.5){$450$};
\node[centered]at (6,-1.5){$502$};
\draw[thick, decorate,decoration={brace, amplitude=12pt,mirror}, xshift=0pt, yshift =0pt] (4,4) -- (6,4) node[ black, midway, below, yshift=-0.3cm]{Escasez};
\end{tikzpicture}
\end{minipage}%
\begin{minipage}{.5\textwidth}
\centering
\begin{eqnarray*}
\textnormal{Dado que } & & 90=\frac{1}{5} q^s_t\\
\textnormal{Por tanto} & &  q^s_t = 450\\
\textnormal{Del mismo modo} & & 90=2600-5q^d_t\\
\textnormal{y} & & q^d_t = 502\\
\end{eqnarray*}
\end{minipage}%

\begin{itemize}
	\item ¿Cuántos kilogramos serán vendidos?
\end{itemize}

Se venderán 450 Kg

\begin{itemize}
\item ¿Es un equilibrio sostenible? Explique
\end{itemize}

No, ya que existe un exceso de demanda que no está siendo suplida por la oferta (escasez) y hay compradores que están dispuestos a pagar más. Suponiendo que el gobierno quitara el precio techo o que no fuese efectivo controlando el cumplimiento de este (ej: Mercado Negro) el precio y la cantidad tenderían al equilibrio en el largo plazo.\\

D.	Si el gobierno impone un impuesto de \$26/kg, en el equilibrio:

\begin{minipage}{.5\textwidth}
	\centering
	\begin{tikzpicture}[scale=0.5]
	\draw[thick,<->] (0,10) node[above]{$P$}--(0,0)--(10,0) node[right]{$Q$}; % Axis and Lable
	\node [below left] at (0,0) {$o$};%Origin
	\draw[red](1,1)--(9,9) node[right]{$S$};
	\draw[blue](1,9)--(9,1) node[right]{$D$};
	\draw[dashed, black!30!green](0,5)--(5,5) node[right]{$$};
	%\draw[dashed, black!30!green]((5,0)--(5,5) node[right]{$$};
	\filldraw[fill=black!30!green] (5,5) circle ( 5pt) {};
	\draw[dashed, black](0,6)--(4, 6) node[right]{$$};
	\draw[dashed,black](4,-1)--(4, 6) node[right]{$$};
	\draw[dashed,black](0,4)--(4, 4) node[right]{$$};
	\node[left]at (0,5){$100$}; %Texto en eje y en linea de equilibrio
	\node[left]at (0,4){$99$}; %Texto en eje y en linea de equilibrio
	\node[left]at (0,6){$125$}; %Texto en eje y en linea de equilibrio
	%\node[centered]at (5,-0.75){$500$}; %Texto en eje x en linea de equilibrio
	\node[centered]at (4,-1.5){$495$};
	
	\draw[thick, decorate,decoration={brace, amplitude=12pt, }, xshift=46pt, yshift =285pt, rotate=270] (4,4) -- (6,4) node[ black, midway, right, xshift=0.4cm, yshift=-0.1cm]{Impuesto = 26};
	\end{tikzpicture}
\end{minipage}%
\begin{minipage}{.5\textwidth}
	\centering
	\begin{eqnarray*}
		\textnormal{Dado que } & & p_d = p_s + Impuesto\\
		\textnormal{Entonces} & &  2600-5q^*  = \frac{1}{5} q^* + 26\\
		\textnormal{Así se tiene que} & & q^T = 495\\
		\textnormal{Por lo que} & & p^d_t = 125 \\
		\textnormal{Y} & & p^s_t = 99 \\
	\end{eqnarray*}
\end{minipage}%

\begin{itemize}
	\item ¿Qué cantidad del producto es vendida?
\end{itemize}
Se venden 495 unidades del producto.
\begin{itemize}
	\item ¿Cuál es el precio final recibido por el vendedor? 
\end{itemize}
El precio final recibido por el vendedor es de \$99
\begin{itemize}
	\item ¿Cuál es el precio pagado por el comprador?
\end{itemize}
El precio final pagado por el comprador es de \$125\\\\	
E.	Si el gobierno impone un impuesto tipo IVA de 26\%, en el equilibrio:\\
\begin{minipage}{.5\textwidth}
	\centering
	\begin{tikzpicture}[scale=0.5]
	\draw[thick,<->] (0,10) node[above]{$P$}--(0,0)--(10,0) node[right]{$Q$}; % Axis and Lable
	\node [below left] at (0,0) {$o$};%Origin
	\draw[red](1,1)--(9,9) node[right]{$S$};
	\draw[blue](1,9)--(9,1) node[right]{$D$};
	\draw[dashed, black!30!green](0,5)--(5,5) node[right]{$$};
	%\draw[dashed, black!30!green]((5,0)--(5,5) node[right]{$$};
	\filldraw[fill=black!30!green] (5,5) circle ( 5pt) {};
	\draw[dashed, black](0,6)--(4, 6) node[right]{$$};
	\draw[dashed,black](4,-1)--(4, 6) node[right]{$$};
	\draw[dashed,black](0,4)--(4, 4) node[right]{$$};
	\node[left]at (0,5){$100$}; %Texto en eje y en linea de equilibrio
	\node[left]at (0,4){$99,0099$}; %Texto en eje y en linea de equilibrio
	\node[left]at (0,6){$124,7525$}; %Texto en eje y en linea de equilibrio
	%\node[centered]at (5,-0.75){$500$}; %Texto en eje x en linea de equilibrio
	\node[centered]at (4,-1.5){$495,0495$};
	
	\draw[thick, decorate,decoration={brace, amplitude=12pt, }, xshift=46pt, yshift =285pt, rotate=270] (4,4) -- (6,4) node[ black, midway, right, xshift=0.4cm, yshift=-0.1cm]{Impuesto = 26\%};
	\end{tikzpicture}
\end{minipage}%
\begin{minipage}{.5\textwidth}
	\centering
	\begin{eqnarray*}
		\textnormal{Dado que } & & p_s * (1+T) = \frac{1 + 0,26}{5}q^T\\
		\textnormal{Entonces} & &  0,252*q^T  = 2600 - 5q^T\\
		\textnormal{Así se tiene que} & & q^T = 495,0495\\
		\textnormal{Y} & & p^d_t = 124.7525 \\
		\textnormal{Por lo que} & & p^s_t = 99,0099 		
	\end{eqnarray*}
\end{minipage}%

\begin{itemize}
	\item ¿Qué cantidad del producto es vendida?
\end{itemize}	
Se venden 495,0495 (o ~495) unidades.
\begin{itemize}
	\item ¿Cuál es el precio final recibido por el vendedor? 
\end{itemize}	
El vendedor recibe un precio de \$99,0099 por unidad o ~\$99 por unidad.
\begin{itemize} 
	\item ¿Cuál es el precio final recibido por el vendedor? 
\end{itemize}	
El vendedor recibe un precio de \$124,7225 por unidad o ~\$125 por unidad.
\end{document}
